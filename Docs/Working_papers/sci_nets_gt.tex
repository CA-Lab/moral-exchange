%%%%%%%%%%%%%%%%%%%%%%% file template.tex %%%%%%%%%%%%%%%%%%%%%%%%%
%
% This is a general template file for the LaTeX package SVJour3
% for Springer journals.          Springer Heidelberg 2010/09/16
%
% Copy it to a new file with a new name and use it as the basis
% for your article. Delete % signs as needed.
%
% This template includes a few options for different layouts and
% content for various journals. Please consult a previous issue of
% your journal as needed.
%
%%%%%%%%%%%%%%%%%%%%%%%%%%%%%%%%%%%%%%%%%%%%%%%%%%%%%%%%%%%%%%%%%%%
%
% First comes an example EPS file -- just ignore it and
% proceed on the \documentclass line
% your LaTeX will extract the file if required
\begin{filecontents*}{example.eps}
%!PS-Adobe-3.0 EPSF-3.0
%%BoundingBox: 19 19 221 221
%%CreationDate: Mon Sep 29 1997
%%Creator: programmed by hand (JK)
%%EndComments
gsave
newpath
  20 20 moveto
  20 220 lineto
  220 220 lineto
  220 20 lineto
closepath
2 setlinewidth
gsave
  .4 setgray fill
grestore
stroke
grestore
\end{filecontents*}
%
\RequirePackage{fix-cm}
%
%\documentclass{svjour3}                     % onecolumn (standard format)
%\documentclass[smallcondensed]{svjour3}     % onecolumn (ditto)
\documentclass[smallextended]{svjour3}       % onecolumn (second format)
%\documentclass[twocolumn]{svjour3}          % twocolumn
%
\smartqed  % flush right qed marks, e.g. at end of proof
%
\usepackage{graphicx}
\usepackage{hyperref}
\usepackage[section]{placeins}
% \usepackage{mathptmx}      % use Times fonts if available on your TeX system
%
% insert here the call for the packages your document requires
%\usepackage{latexsym}
% etc.
%
% please place your own definitions here and don't use \def but
% \newcommand{}{}
%
% Insert the name of "your journal" with
% \journalname{myjournal}
%
\begin{document}

\title{A game-theory modeling approach to fitness and trust dynamics in biomedical research social networks}%\thanks{Grants or other notes
%about the article that should go on the front page should be
%placed here. General acknowledgments should be placed at the end of the article.}

%\subtitle{Do you have a subtitle?\\ If so, write it here}

\titlerunning{Fitness \& trust in biomedical research networks}        % if too long for running head

\author{J. Mario Siqueiros-Garc\'ia         \and      Rodrigo Garc\'ia-Herrera \and         Enrique Hern\'andez-Lemus \and
        Sergio Alcal\'a  %etc.
}

%\authorrunning{Short form of author list} % if too long for running head

\institute{J. Mario Siqueiros-Garc\'ia \at
  IIMAS-UNAM. Circuito Escolar 3000, Cd Universitaria, Coyoacán, Ciudad de México, D.F. \\
              Tel.: +123-45-678910\\
              Fax: +123-45-678910\\
              \email{jmario.siqueiros@iimas.unam.mx}           %  \\
%             \emph{Present address:} of F. Author  %  if needed
              \and
              Rodrigo Garc\'ia-Herrera,\\ Enrique Hern\'andez-Lemus, \\ Sergio Alcal\'a-Corona \at
              National Institute of Genomic Medicine\\
              Perif\'erico Sur 4809, Arenal Tepepan, Tlalpan, 14610, \\
              Ciudad de M\'exico, D.F., M\'exico  
           %% S. Author \at
           %%    second address
}

\date{Received: date / Accepted: date}
% The correct dates will be entered by the editor


\maketitle

\begin{abstract}
\textbf{Background:} In an ideal world, access to research resources
should be fair and equitative according to the proposals relevance and
the researcher's academic record. We know that this is not necessarily
so, specially in places where access to some resources, e.g.,
biological samples, is not regulated. Other factors may come into play
like social connections, political power or prestige. In this work we
explore the distribution of fitness and trust in a biomedical
researchers collaboration network when playing a variation of an
iterative prisoner's dilemma in which agents are compromised in either
defecting and increasing their individual fitness or cooperating and
increase mutual fitness with their neighbors. \textbf{Methods:}
Fitness is a property of the each agent and trust is a property of the
link between two agents. According to a pay-off matrix and a mutual
trust $A_{ij}$ matrix, we get a measure of distrust or confidence
for each node. If the agents' confidence is below certain value then
the agent will act suspiciously and will defect, othewise it will
cooperate. We tested our simulation on an Erd\"{o}s-Reny\'i, a
Watts-Strogatz small-world and Barab\'asi-Albert topologies, as well
as on a real biomedical research network. Agents behavior is updated
in a synchronous manner. \textbf{Results:} All networks find a point
of equilibrium before the $50^{th}$ iteration. Different topologies
display different fitness and trust distribution. Fitness in an
Erd\"{o}s-Reny\'i netwok follows a normal distribution and trust is
bimodal. For a Watts-Strogatz, small world networks, both fitness and
trust distributions are strongly skewed to the right.
Barab\'asi-Albert topology has a heavy left-skewed distribution
(resembling to a power-law) and trust is bimodal. The biomedical
researchers network has fitness distribution as in a
Barab\'asi-Albert, but trust is distributed as in a Watts-Strogatz,
small world topology. \textbf{Discussion:} 1) The distribution of
fitness in the researchers network suggests that there are mechanisms
governing the network that produces an asymetric access to resources.
2) Nevetheless, trust variables behaves as in the small-world model
which might reflect some sort of subordination among researchers by
which they are obliged to cooperate. 3) The range in the threshold
that regulates the stringency to cooperate or defect according to the
agent's distrust is small, suggesting that there is a region in
which a phase transition takes place from a population full of
defectors to a population of cooperators 4) Finally, we would like to
propose that this sort of work may help to make visible the need to
develop science policies to promote a better, small world-like, fair
fitness distribution.  

%% Insert your abstract here. Include keywords, PACS and mathematical
%% subject classification numbers as needed.
\keywords{Game theory \and Trust \and Models \and Biomedical research community \and Social Networks}
% \PACS{PACS code1 \and PACS code2 \and more}
% \subclass{MSC code1 \and MSC code2 \and more}
\end{abstract}

\section{Introduction}
\label{intro}

Collaboration has become a corner stone in scientific research today.
Though physics has a long history and experience in collaborative
projects, biology has recently become an evermore collaborative
discipline\cite{Vermeulen2013}. Biology has interesting record in such
matters because scientific collaboration has been perceived
differently depending on the branch of biology one is taking about. As
molecular biology has traditionally been a research activity of small
laboratories\cite{KnorrCetina1999,Strasser2006}. On the extreme, there
is natural history exchanging data and samples since the $XVII^{th}$
century\cite{Muller2012,Strasser2012}.\\ 

Despite the differences in culture and practices, the Human Genome
Project made collaboration a central feature of biology. Now a day it
is widely aknowledge that collaboration takes many forms, from
biological samples and biobanking to international groups in charge of
helping the bio-research communities to harmonize and share their
data.\\ 

As has been pointed by others, social research on scientific
collaboration is mostly about physics and space research\cite{Vermeulen2013}.
Collaboration in biology has been studied to a lesser degree. The
majority of work done on the subject has explored the quantiative
aspects of it\cite{Newman2001,Newman2004,Elango2012,HernandezLemus2013} and not so much the
qualitative aspects of collaboration \cite{Strasser2006,Strasser2012}.\\


As a quantitative methodology, simulating processes in science has its own history
\cite{Gilbert1997,Edmonds2011}, but as in most social sciences, simulation has 
been a tangential methodology. In this article, we explore the dynamics of
trust and fitness (resources) distribution by means of a simulation.
This approach is primarily quantitative, nevertheless and to a certain
degree, the idea behind the model is ethnographically informed.
Inspired by game theory models we implemented our simulation on a 
real social network of biomedical researchers from M\'exico. Our main objective is
to achieve a better understanding of the role of network topology on
social dynamics. The dynamics of the system are explored following the
distribution of fitness --as a proxy for resources-- and trust as
agents on the network play the prisoner's dilemma while constrained by
trust with their neighbors.\\ 


Game theoretical approaches to cooperation have –of course– been
studied for decades. However, taking into account the nature of social
interaction among players by means of an undelying social network is a
more recent area of research. Some 10 years ago, Lo and colleagues
\cite{Lo2004} formulated a minority game model to connected population
within a network and study the role of connectivity on such games.
Later on, Buesser and Tomassini \cite{Buesser2012} studied the
dynamics of cooperation for players located on a spatially embedded
network in order to categorize the effect that different interaction
structures may have in maintaning cooperation (or defection) domains.\\

The impact that network topology may have in the robustness of
cooperation was investigated in numerical experiments and simulations
\cite{Ichinose2013} revealing that clusters of cooperating agents become robust by
increasing the number of links to cooperators, with the possible
implication that a large number of individuals is required to maintain
global cooperation patterns.\\

Regarding possible strategies to improve cooperation patterns, several mech-
anisms such as inequity aversion \cite{Ahmed2014}, social imitation
and strategic choice \cite{Vilone2014} have been explored. Inequity
aversion is the condition in which individuals care about payoff
equality in the outcomes. A kind of moral memory in past outcomes
implies that inequity aversion promotes cooperation
\cite{Ahmed2014}. Social imitation contrasts with strategic choices
in such a way that social influence models may impact on background
rationality in decision making. This balance mey be behind moody
conditional cooperation within a social environment \cite{Vilone2014}.\\

The analysis of cooperation in scientific research has been also the subject
of a number of studies. This is not suprising since cooperation and competi-
tion are quite important in today’s academic success. How do collaboration
happens within a competitive academic environment and what kind of payoff
is present in these settings were questions considered recently by Wardil and
Hauert \cite{Wardil2015} in the context of cooperation in multiauthored
publications.\\

In spite of all these research efforts, cooperation in the context of scientific
collaboration is still loosely defined and the long time dynamics of academic
cooperation (and its consequences) are yet to be fully disclosed. In this context
our work plans to contribute to further our current understanding on the
matter.

The manuscript is structured in four sections. The following section we will present \textit{FOSISS}, the main program for grants destined for biomedical applied research in M\'exico. To describe \textit{FOSISS} is important because that is the source of our database for creating the researchers collaboration network. In the next section we describe our model, we also introduce the different network topologies on which we explored the model. In the fourth part of the manuscritp we present the results and discuss them. In the last part we will draw some final remarks.

\section{Biomedical research: CONACyT and FOSISS}
\label{sec:1}
CONACyT (Council of Science and Technology) is the Mexican government
entity in charge of promoting the development of science and 
technology. It is also in charge of creating the country's scientific
policies. In a way, CONACyT is equivalent to the National Science
Foundation of the United States.\\

Among CONACyT's functions it is to develop science and technology
policies according to national needs and demands, to offer
assessorship to the different instances of the government on
scientific and technological topics, to promote the creation of
research networks among the scientific community, grant scholarships
for masters and doctoral studies, and it also manages different trusts
intended to fund individuals and groups for scientific and
technological research.\\

In the year 2002 CONACyT, along with other government agencies and
entities have created sectorial funds or \emph{Fondos Sectoriales}.
Sectorial funds are trusts for scientific and technological research.
The objective of such funds is to cover and equally promote the
research capacities of different areas such as energy, agriculture, or
technological innovation by means of the generation of human resources
and helping research groups to consolidate. It is expected that the
knowledge generated under the sponsorship of \emph{Fondos Sectoriales}
to be the product of applied research that attends national public
needs, and promotes economic growth.\\

\emph{FOSISS} or Sectorial Fund for Health and Social Security
Research (\emph{Fondo Sectorial en Investigaci\'on en Salud y
  Seguridad Social}) is among such funds. FOSISS is constituted by
CONACyT, SSA, IMSS and ISSSTE,\footnote{SSA is the acronym for
  Secretariat of Health \emph{Secretar\'ia de Salud}; IMSS is the
  acronym for Social Security Mexican Institute (\emph{Instituto
    Mexicano del Seguro Social}); ISSSTE stands for Institute for
  Social Security and Services for State Workers (\emph{Instituto de
    Seguridad y Servicios Sociales de los Trabajadores del Estado})}
being all of them the major public health providers and research
institutions in the country. Every year CONACyT opens a call for funds
limited to a set the health areas previously defined by a group of
experts. Such areas range from public health issues to chronic and
degenerative diseases.\\

Elegibility is open to the public and private health research
sectors, nevertheless, most applicants are public universities and
research institutions. To 2012, there were 91 institution who have
been part of the funding program and comprise since then 4122
researchers.

\section{Methodology}
\label{sec:2}
In this work we developed a model in order to have a better
understanding the distribution of fitness and trust in a biomedical
researchers collaboration under certain social constrictions or
topological properties of real collaborative networks of biomedical
research.\\

Our model is based on the iterative version of the prisoner's dilemma (PD)
instantiated on networks. Implementing games on networks is not new
and is an active area of research aimed to understand the evolution of
cooperation in networks populated by selfish agents
\cite{Szabo2007,Nowak1992,Nowak2006}. In many network models on which some of game
theory games are simulated, agents decision to cooperate or defect
depends on a specific strategy, as the well known \textit{tit-for-tat}
\cite{Axelrod2006,Nowak2011}. In some other cases, agents can modify the weight
of the interactions with their neighbors \cite{Santos2006}. From a different perspective
others have explored the effect of different topologies on the
emergence of cooperation \cite{Santos2005,Hauert2004}. In our case, agents decision
to cooperate or defect is a probabilistic outcome that depends on the
agent's confidence. Confidence, as it turns out, depends on how
positive the agent's gains and relations towards its neighbors
have been in the past and we study the behavior of the system under
different topologies, including a real-world network.\\ 

In the model that we developed, agents are embedded in a network with
varying number of neighbors. Following the traditional PD game, the
strategy chosen by an agent and the strategy chosen by its neighbors
will produce a pay-off. Pay-off follows the rule: $T > R > P > S$. $T$
is for temptation to defect. It is the highest pay-off and it takes
place when the player defects and the other cooperates. $R$ is for
reward for when both players cooperate. $P$ is the punishment for when
both players defect. And $S$ is for suckers pay-off. Fitness is a
property of agents in which pay-off is accumulated.\\ 

{\bf Prisoner's Dilemma fitness pay-off matrix}\\

\begin{tabular}{| l | l | l |}
\hline
          & \bf{Cooperate} & \bf{Defect} \\ \hline
\bf{Cooperate} &  $R,R$      &  $S,T$   \\ \hline
\bf{Defect}    &  $T,S$      &  $P,P$   \\ \hline

\end{tabular}\\ \\

Trust is a property of the link between two agents and it is updated
according to a mutual trust $A_{ij}$ matrix. In the trust matrix, the
highest value goes to an edge when both agents cooperate, getting an
$R$ for reward, if one of them defects, trust is negatively affected
or $P$ for the trust being punished. If both agents defect, trust
value doesn't change, which in a sense, it means that agents didn't
interact or that the interaction gets nullified $N$. \\

{\bf Trust pay-off matrix}\\

\begin{tabular}{| l | l | l |}
\hline
          & \bf{Cooperate} & \bf{Defect} \\ \hline
\bf{Cooperate} &  $R$      &  $P$   \\ \hline
\bf{Defect}    &  $P$      &  $N$   \\ \hline

\end{tabular}\\ \\


After each game, the agent adds-up fitness ($w$), which is the sum of the
pay-offs following the PD matrix. The agent also adds-up the amount of
trust ($t_{ji}$) that shares with its neighbors. We measure global
fitness and trust for the whole network. Global fitness is the sum of
all individual fitness and trust is the sum of every pair of agents
links trust. In order for the agent to choose to cooperate
or defect, it is assigned a degree of credulity. Credulity is
calculated as the sum of the agent's normalized averages of fitness
and trust over the number of the agent's neighbors.\\


\textit{credulity} $= \langle w \rangle +  \langle t_{ji} \rangle$\\

If the agents' credulity is below certain global threshold of distrust, then
the agent will act suspiciously and will defect, othewise it will
cooperate. We tested our simulation on an Erd\"{o}s-Reny\'i, a
Watts-Strogatz small-world and Barab\'asi-Albert topologies, as well
as on a real biomedical research network. Agents distrust, state,
fitness and edge's trust is updated in a synchronous manner.   

\subsection{Implementation of the model in different topologies}

\subsubsection{Erd\"{o}s-R\'enyi}

\subsubsection{Small-World}

\subsubsection{Barb\'asi-Albert}

\subsubsection{Biomedical research community network}

The biomedical research network on which we are running our model
was generated with data from collaborative projects. Our data was
obtained from CONACyT and includes information for twelve years of
\textit{FOSISS} grants. Data included names of Principal
Investigators, collaborators, research topics, etc. The network we are
using here has researchers as nodes and edges represent the connection
of two scientists when they participate in the same project. Edges are
also weighted according to the number of projects shared by the pair
of scientists. The complete network summed-up a total of 145
components or subnetworks, but we are running the model on the only
giant component made-up of 4122 researchers, and 23391 edges.\\

The giant component is a well integrated network, with a clustering
coefficient $\langle C \rangle = 0.870$, an average shortest path
length of $\langle l \rangle = 5.493$ and a density of $p = 0.003$. Such
properties recall a small-world topology \cite{Watts1998}, and a great
deal of self-organization when compared to a random network with the
same density and number of nodes.\footnote{Using networkx in Python we
created a null model to test such properties of our network on a
random graph. The results were the expected with a $\langle C
\rangle = 0.00279$, and a $\langle l \rangle = 3.6$
\cite{Watts1998}.} The network centralization is $0.023$, since there
are no visible researchers that play as hubs in the network.
Nevertheless the network heterogeneity is $0.873$, which means that
the network is highly hierarchical. When the degree distribution is
analyzed, degree deceases as a power-law with an exponent of $1.7$,
similar to other social networks described as scale-free topology
networks \cite{Barabasi1999}. Finally, the number of neighbors of each
node is $11.39$.\\  



\section{Results and Discussions}
\label{sec:3}

There are several results to report. Among the most relevant, we found that between $0.19$ and $0.24$ of the \textit{distrust} parameter, there is a phase transition in all different topologies and for the different variables. Nevertheless, it is worth noticing that the phase transition is different accoring to the topology of the network at stake. When \textit{distrust} is between $0.2$ and $0.0$, that is, when there is no space for suspiciousness, all agents cooperate, when \textit{distrust} is above $0.25$, all agents defect. \textit{Fitness}, \textit{trust}, and \textit{changing state population} replicate that  same behavior for the same limits. \\

Another result, central to our argument is the differences in \textit{fitness} and \textit{trust} distribution at the end of the simulation, for every topology. \textit{Fitness} distribution is similar to the \textit{degree} distribution of each topologoy. Most of all, we found that \textit{fitness} distribution for the \textit{FOSISS} network, resambles quite accurately to the distribution of \textit{fitness} in the \textit{Barab\'asi-Albert} network.

\begin{figure}
% Use the relevant command to insert your figure file.
% For example, with the graphicx package use
%\includegraphics[scale=0.6]{DC.pdf}
% figure caption is below the figure
\caption{Evolution of the proportion of cooperator/defector ratio as a function of time and \emph{distrust} for different network topologies. Panel A: Erd\"os-Reny\'i, Panel B: Watts-Strogatz, Panel C: Barabasi-Alberts, Panel D: Real FOSISS network.}
\label{fig:3}       % Give a unique label
\end{figure}

\FloatBarrier

\begin{figure}
% Use the relevant command to insert your figure file.
% For example, with the graphicx package use
%\includegraphics[scale=0.6]{Fitness.pdf}
% figure caption is below the figure
\caption{Evolution of fitness as a function of time and \emph{distrust} for different network topologies. Panel A: Erd\"os-Reny\'i, Panel B: Watts-Strogatz, Panel C: Barabasi-Alberts, Panel D: Real FOSISS network.}
\label{fig:2}       % Give a unique label
\end{figure}

\FloatBarrier

\begin{figure}
% Use the relevant command to insert your figure file.
% For example, with the graphicx package use
%\includegraphics[scale=0.6]{Trust.pdf}
% figure caption is below the figure
\caption{Evolution of trust as a function of time and \emph{distrust} for different network topologies. Panel A: Erd\"os-Reny\'i, Panel B: Watts-Strogatz, Panel C: Barabasi-Alberts, Panel D: Real FOSISS network.}
\label{fig:1}       % Give a unique label
\end{figure}

\FloatBarrier


\begin{figure}
% Use the relevant command to insert your figure file.
% For example, with the graphicx package use
%\includegraphics[scale=0.6]{State.pdf}
% figure caption is below the figure
\caption{Evolution of the rate of state (cooperator/defector) flipping as a function of time and \emph{distrust} for different network topologies. Panel A: Erd\"os-Reny\'i, Panel B: Watts-Strogatz, Panel C: Barabasi-Alberts, Panel D: Real FOSISS network.}
\label{fig:4}       % Give a unique label
\end{figure}

\FloatBarrier



\section{Discussion}
\label{sec:4}




%% Text with citations \cite{RefB} and \cite{RefJ}.
%% \subsection{Subsection title}
%% \label{sec:2}
%% as required. Don't forget to give each section
%% and subsection a unique label (see Sect.~\ref{sec:1}).
%% \paragraph{Paragraph headings} Use paragraph headings as needed.
%% \begin{equation}
%% a^2+b^2=c^2
%% \end{equation}

% For one-column wide figures use
%%\begin{figure}
% Use the relevant command to insert your figure file.
% For example, with the graphicx package use
%%  \includegraphics{example.eps}
% figure caption is below the figure
%%\caption{Please write your figure caption here}
%%\label{fig:1}       % Give a unique label
%\end{figure}
%
% For two-column wide figures use
%\begin{figure*}
% Use the relevant command to insert your figure file.
% For example, with the graphicx package use
%  \includegraphics[width=0.75\textwidth]{example.eps}
% figure caption is below the figure
%\caption{Please write your figure caption here}
%\label{fig:2}       % Give a unique label
%\end{figure*}
%
% For tables use
%\begin{table}
% table caption is above the table
%\caption{Please write your table caption here}
%\label{tab:1}       % Give a unique label
% For LaTeX tables use
%\begin{tabular}{lll}
%\hline\noalign{\smallskip}
%first & second & third  \\
%\noalign{\smallskip}\hline\noalign{\smallskip}
%number & number & number \\
%number & number & number \\
%\noalign{\smallskip}\hline
%\end{tabular}
%\end{table}


\begin{acknowledgements}
The authors gratefully acknowledge support by grants:
  CB-222220-R/2013 and CB-179431/2012 (Consejo Nacional de Ciencia y
  Tecnolog\'ia). We would also like to aknowledge CONACyT for letting
  heve access to their database. Finally, we aknowledge the help of Francisco
  Allende for his work cleaning researchers databses.

%If you'd like to thank anyone, place your comments here
%and remove the percent signs.
\end{acknowledgements}

% BibTeX users please use one of
%\bibliographystyle{spbasic}      % basic style, author-year citations
%\bibliographystyle{spmpsci}      % mathematics and physical sciences
%\bibliographystyle{spphys}       % APS-like style for physics
%\bibliography{}   % name your BibTeX data base

% Non-BibTeX users please use
\begin{thebibliography}{50}

\bibitem{Vermeulen2013} Vermeulen, Nikki; Parker, John N \& Penders,
  Bart, Understanding life together: A brief history of collaboration
  in biology, \textit{Endeavour}, 37(3): 162-171 (2013) 

\bibitem{KnorrCetina1999} Knorr-Cetina, Karol; \textit{Epistemic
  cultures: how the sciences make knowledge}, MIT Press, Cambridge, MA
  (1999) 

\bibitem{Strasser2006} Strasser, B.J. Collecting and Experimenting:
  The Moral Economies of Biological Research, 1960s-1980s.
  \textit{Preprint no. 310.} Berlin: Max Planck Institute for the
  History of Science, (2006) 

\bibitem{Muller2012} M\"{u}ller-Wille, S. \& Charmantier. I.  Natural
  history and information overload: The case of Linnaeus.
  \textit{Studies in History and Philosophy of Biological and
    Biomedical Sciences.} 43:4-15, (2012) 

\bibitem{Strasser2012} Strasser, BJ. Data-driven sciences: From
  wonder cabinets to electronic databases. \textit{Studies in History
    and Philosophy of Biological and Biomedical Sciences} 43:85-87,
  (2012) 

\bibitem{Newman2001} Newman, MEJ. Clustering and preferential
  attachment in growing networks. \textit{Physical Review E},
  64:025102-1/025102-4, (2001) 

\bibitem{Newman2004} Newman, MEJ. Coauthorship networks and patterns
  of scientific collaboration. \textit{PNAS}, 101, no. suppl 1,
  5200-5205 (2004) 

\bibitem{Elango2012} Elango, B. \& Rajendran, J. Authorship Trends and
  Collaboration Pattern in the Marine Sciences Literature: A
  Scientometric Study. \textit{International Journal of Information
    Dissemination and Technology}, 2(3), 166-169, (2012)
  
  
\bibitem{HernandezLemus2013} Hern\'andez-Lemus, Enrique \&
  Siqueiros-Garc\'ia JM, Information theoretical methods for complex
  network structure reconstruction. \textit{Complex Adaptive Systems
    Modeling}. 1: 8, doi:10.1186/2194-3206-1-8 (2013).

\bibitem{Gilbert1997} Gilbert, N. A Simulation of the Structure of
  Academic Science. \textit{Sociological Research Online}, 2(2) 3,
  http://www.socresonline.org.uk/2/2/3.html 
  
\bibitem{Edmonds2011} Edmonds, B., Gilbert, N., Ahrweiler, P., \&
  Scharnhorst, A. Simulating the Social Processes of Science.
  \textit{Journal of Artificial Societies and Social Simulation},
  14(4) 14,  http://www.jasss.soc.surrey.ac.uk/14/4/14.html
  
\bibitem{Lo2004} Lo, T.S., Chan, H.Y., Hui, P.M., Johnson, N.F.,
  Theory of networked minority games based on strategy pattern
  dynamics, \textit{Physical Review E} 70, 056012 (2004) 

\bibitem{Buesser2012} Buesser, P. and Tomassini, M., Evolution of
  cooperation on spatially embedded networks, \textit{Physical Review E} 86,
  066107 (2012)  

\bibitem{Ichinose2013} Ichinose, G., Tenguishi, Y., and Tanizawa, T.,
  Robustness of cooperation on scale-free networks under continuous
  topological change, \textit{Physical Review E} 88, 052808 (2013) 

\bibitem{Ahmed2014} Ahmed, A., Karlapalem, K., Inequity aversion and
  the evolution of cooperation, \textit{Physical Review E} 89, 022802, (2014)

\bibitem{Vilone2014} Vilone, D., Ramasco, J.J., S\'anchez, A., San
  Miguel, M., Social imitation versus strategic choice, or consensus
  versus cooperation, in the networked Prisoner’s Dilemma,
  \textit{Physical Review E} 90, 022810, (2014) 
 
\bibitem{Wardil2015} Wardil, L., Hauert, C., Cooperation and coauthorship in
  scientific publishing, Physical Review E 91, 012825, (2015)
  
\bibitem{Szabo2007} Szab\'o, Gy\"{o}rgy \& F\'ath, G\'abor,
  Evolutionary games on graphs. \textit{Physics Reports}, 446: 97-216
  (2007) 

\bibitem{Nowak1992} Nowak, Martin A. \& May, Robert M, Evolutionary
  Games and Spatial Chaos. \textit{Nature}, 359: 826-829, (1992)

\bibitem{Nowak2006} Oshtuki, Hisashi, Hauert, Christoph, Lieberman, Erez \& Nowak, Martin A, A simple rule for the evolution of cooperation on social graphs. \textit{Nature}, 441: 502-505 (2006)

\bibitem{Axelrod2006} Axelrod, Robert, \textit{Evolution of cooperation: Revised edition}, Basic Books, Cambridge, MA (2006)

\bibitem{Nowak2011} Nowak, Martin A. \& Highfield, Roger,
  \textit{Supercooperators. Altruism, Evolution, and Why We Need Each
    Other to Succeed}. Free Press, New York, NY  (2011)

\bibitem{Santos2006} Santos, Francisco C., Pacheco, Jorge M., Lenaerts, Tom, Cooperation prevails when individuals adjust their social ties. \textit{PLoS Computational Biology}. 2(10): 1284-1291 (2006)

\bibitem{Santos2005} Santos, F.C. \& Pacheco, J.M., Scale-Free
  Networks Provide a Unifying Framework for the Emergence of
  Cooperation. \textit{PRL}, 95: 098104-1 to 098104-4, (2005) 

\bibitem{Hauert2004} Hauert, Christoph \& Doebeli, Michael, Spatial
  structure often inhibits the evolution of cooperation in the
  snowdrift game. \textit{Nature}, 428: 643-646,  (2004)
  
\bibitem{Watts1998} Watts, Duncan J. \& Strogatz, Steven H, Collective dynamics
  of 'small-world' networks. \textit{Nature}, 393: 440-442 (1998)

\bibitem{Barabasi1999} Barab\'asi, Albert-L\'aszl\'o. \& Albert, R\'eka, Emergence of
  Scaling in Random Networks. \textit{Science}, 286: 509-512 (1999)


%%Discusion references

\bibitem{Merton1968} Merton, Robert K, The Matthew Effect in Science, \textit{Science}, 159(3810): 56-63 (1968)


% and use \bibitem to create references. Consult the Instructions
% for authors for reference list style.
%
%% \bibitem{RefJ}
%% % Format for Journal Reference
%% Author, Article title, Journal, Volume, page numbers (year)
%% % Format for books
%% \bibitem{RefB}
%% Author, Book title, page numbers. Publisher, place (year)
%% % etc
\end{thebibliography}

\end{document}
% end of file template.tex

